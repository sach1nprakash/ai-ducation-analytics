\section{Introduction}
  Coasters prevent condensation from dripping along the glass, which can damage the surface of tables. The project referred to as \textbf {CHEERS} deals with two circular coasters overlapping each other. The objective of \textbf {CHEERS} is to compute the diameter of the overlapping region such that 
  the real estate is half that of any of the coasters.

\section{Background}
  In this project, we determine the roots of an equation to calculate the angle made at the vertex of a coaster using the \textbf{Secant approximation method}. This method approximates the value of a function using a secant line passing through two points on a graph.
  $$ x_{n+1} = x_n - \frac{f(x_n) \cdot (x_n - x_{n-1})}{f(x_n) - f(x_{n-1})} $$

  \vspace*{20pt}
  \noindent The Sin/Cos value is approximated using the \textbf{McLaurin Series} which is the sum of derivatives of a function.
  $$\sin(x) = \sum_{n=0}^{\infty} \frac{(-1)^n}{(2n+1)!}x^{2n+1}$$
  $$\cos(x) = \sum_{n=0}^{\infty} \frac{(-1)^n}{(2n)!}x^{2n}$$ 

\section{Scope}
  The length \textbf{l} of the overlapping coaster region can be computed by the equation
  $$l = 2R\left(1 - \cos\frac{\alpha}{2}\right)$$

  \indent R \textrightarrow \;radius of the coaster \\
  \indent $\alpha$ \textrightarrow \;angle created with the vertex at the centre point of the left coaster

  \vspace{20pt}
  The angle, $\alpha$ can be calculated using
  $$\alpha - \sin(\alpha) = \frac{\pi}{2}$$
  
  \vspace*{20pt}
  
  \begin{itemize}
    \item {The roots of the equation need to be determined to compute the value of $\alpha$. We use the Secant approximation method to arrive at a solution.}
    \item {The McLaurin series has been used to approximate the Sine and Cos values and, subsequently, calculate l.}
  \end{itemize}

\section{Objectives}
\begin{itemize}
  \item To compute the value of $\Pi$.
  \item To calculate the exponent of a number using iteration.
  \item To figure out the factorial of a number using iteration.
  \item To approximate Sin and Cos using McLaurin Series.
  \item To find the roots of an equation using Secant Approximation.
  \item To generate XML files valid with respect to a DTD.
\end{itemize}

\section{Assumptions}
\begin{itemize}
  \item The initial guess values for approximating the roots of an equation will be pre-determined.
  \item The default precision for approximating Sine and Cos functions has been set to 4.
  \item A maximum of 100 attempts will be made to find the roots of the equation, after which the function will be considered non-convergent.  
\end{itemize}
